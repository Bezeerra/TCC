
Este trabalho traz um projeto de implementação de uma nuvem privada no Instituto Federal de Brasília (IFB) utilizando a plataforma open-source OpenStack, com foco na implantação de um \textit{Cluster} modular por meio do \textit{Kolla-Ansible}, que efetuar a conteinerização dos serviços do OpenStack. A arquitetura organiza os nós em funções dedicadas para \textit{Compute}, \textit{Storage} e \textit{Controller}, garantindo alocação eficiente de recursos. Embora limitações de hardware tenham impedido a implementação completa, um ambiente de teste demonstrou com sucesso a integração dos serviços e o potencial de escalabilidade do \textit{cluster} com recursos adicionais no campus de Taguatinga.
    
O projeto utiliza três nós: um nó principal que executa todos os serviços e faz o controle de rede e dois nós especializados para grupos de serviços específicos. Esse design simplifica a adição de novos nós ao \textit{cluster}, assegurando isolamento e controle dos dados. Além disso, também mostramos a integração do \textit{Hierarchical Multitenancy} no OpenStack proporciona isolamento lógico avançado entre projetos para possíveis novas pesquisas, e os \textit{templates HEAT} para a automação de infraestrutura como código (IaC), promovendo maior organização e gestão de recursos.
    
O estudo destaca a viabilidade de, em um futuro próximo, implementar uma nuvem privada no IFB, o que melhoraria o controle da infraestrutura e forneceria recursos valiosos para as pesquisas dos alunos. Essa solução não apenas atende às possíveis necessidades institucionais, mas também estabelece uma base escalável para crescimento futuro, alinhando-se às demandas de ambientes educacionais.
 

\begin{keywords}
Computação em Nuvem, Arquitetura de Nós, OpenStack, Nuvem privada, Virtualização, Kolla-Ansible
\end{keywords}