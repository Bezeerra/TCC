\chapter{Revisão da Literatura}

Com o objetivo de realizar um levantamento de dados sobre o contexto do desenvolvimento de uma nuvem privada, foi aplicado o método de mapeamento sistemático, conforme as diretrizes propostas por \cite{kitchenham2010systematic}.

\section{Mapeamento Sistemático}

O objetivo deste Mapeamento Sistemático é identificar qual o melhor serviço para implementar uma nuvem privada no campus do Instituto Federal de Brasília (IFB). Para responder a essa questão, será realizada uma busca por dados comparativos entre os principais serviços disponíveis atualmente, incluindo: OpenNebula, Apache CloudStack, RedHat OpenShift e OpenStack. A escolha desses serviços foi baseada em sua popularidade no GitHub, no engajamento de suas comunidades ativas e no suporte oferecido a tecnologias modernas. Esses critérios serão explorados detalhadamente ao longo do texto.

\subsection{Metodologia}

A metodologia adotada envolve a comparação de diferentes estudos e artigos que destacam as vantagens e desvantagens de cada serviço. Além disso, serão analisadas as tecnologias e funcionalidades oferecidas por cada solução, visando identificar qual delas melhor atende aos requisitos do ambiente do IFB.

\subsection{Perguntas de Pesquisa}

As seguintes perguntas de pesquisa foram formuladas para orientar este estudo:

\begin{enumerate}
    \item \textbf{RQ1}: As instituições de ensino superior possuem nuvens privadas? Se sim, quais serviços são utilizados?
    \item \textbf{RQ2}: Qual serviço oferece um ambiente mais configurável e escalável?
    \item \textbf{RQ3}: Como as plataformas OpenStack, OpenNebula, Apache CloudStack e OpenShift se comparam em termos de funcionalidades, escalabilidade, segurança e suporte comunitário para atender às necessidades?
\end{enumerate}

\subsection{Critérios de Exclusão}

Os seguintes critérios de exclusão foram estabelecidos para a seleção dos artigos:

\begin{itemize}
    \item \textbf{Falta de relevância}: Artigos que não abordem diretamente a comparação entre diferentes plataformas de nuvem privada, ou que tratem de temas fora do escopo da pesquisa, como nuvens públicas ou híbridas, serão excluídos.
    \item \textbf{Falta de evidências empíricas}: Estudos que não apresentem dados empíricos ou práticos para suportar suas conclusões, como análises baseadas apenas em teoria ou opiniões, serão considerados irrelevantes para o propósito da comparação.
    \item \textbf{Data de publicação}: Artigos publicados antes de 2012 serão excluídos, visto que tecnologias de nuvem evoluíram significativamente a partir desse período, e estudos mais antigos podem não refletir o estado atual das plataformas.
    \item \textbf{Escopo geográfico ou institucional limitado}: Artigos focados em implementações muito específicas de nuvem privada, que não apresentem generalizações aplicáveis para o contexto acadêmico ou institucional de grande porte, serão excluídos.
    \item \textbf{Metodologia não definida}: Artigos que não apresentem uma metodologia clara para a comparação de plataformas, ou que falhem em descrever seus métodos de coleta de dados e análise, serão excluídos por não atenderem aos padrões de rigor científico.
    \item \textbf{Foco em outras tecnologias}: Artigos cujo foco principal seja em outras tecnologias de virtualização ou automação, e não diretamente em plataformas de nuvem privada como OpenStack, serão excluídos.
    \item \textbf{Indisponibilidade de texto completo}: Artigos cujo texto completo não esteja disponível para consulta ou que apenas apresentem resumos ou resenhas sem detalhes suficientes para uma avaliação aprofundada.
\end{itemize}

\section{Análise dos Artigos Selecionados}

\cite{vogel2016private} apresentam uma comparação abrangente dos serviços disponíveis nas principais soluções de Infraestrutura como Serviço \textit{(IaaS)} e Plataforma como Serviço \textit{(PaaS)} atualmente utilizadas: OpenNebula, CloudStack e OpenStack. Adicionalmente, incluímos o OpenShift na comparação, conforme documentado em \cite{OpenShiftDocumentation}. A Tabela \ref{tab:comparacao_ferramentas_nuvem} detalha as funcionalidades e recursos oferecidos por cada plataforma em diferentes camadas, como abstração de recursos, serviço principal, suporte, gerenciamento, segurança, controle e serviços de valor agregado.

Essa comparação tem como objetivo auxiliar na compreensão das diferenças e semelhanças entre essa.s soluções, permitindo uma avaliação mais informada na seleção da plataforma que melhor atenda às necessidades específicas de implementação, escalabilidade, segurança e gerenciamento em ambientes de nuvem.

\small % Reduz o tamanho da fonte para a tabela

\begin{longtable}{|p{3cm}|p{2.8cm}|p{2.8cm}|p{2.8cm}|p{2.8cm}|}
\caption{Comparação de Ferramentas de Nuvem}
\label{tab:comparacao_ferramentas_nuvem} \\
\hline
\textbf{Camada} & \textbf{OpenNebula} & \textbf{CloudStack} & \textbf{OpenStack} & \textbf{OpenShift} \\
\hline
\endfirsthead
\multicolumn{5}{c}%
{\tablename\ \thetable\ -- continuação da página anterior} \\
\hline
\textbf{Camada} & \textbf{OpenNebula} & \textbf{CloudStack} & \textbf{OpenStack} & \textbf{OpenShift} \\
\hline
\endhead
\hline \multicolumn{5}{|r|}{{Continua na próxima página}} \\ \hline
\endfoot
\hline
\endlastfoot
\multicolumn{5}{|c|}{\textbf{Abstração de Recursos}} \\
\hline
Computação & Oned & Libcloud & Nova & Kubernetes \\
\hline
Armazenamento & Interno & Interno & Swift/Cinder & Persistent Volumes \\
\hline
Volume & Interno & Interno & Nova-Volume & Kubernetes Volumes \\
\hline
Rede & Virtual Network Manager & Interno & Neutron/Nova Network & Kubernetes Networking \\
\hline
\multicolumn{5}{|c|}{\textbf{Serviço Principal}} \\
\hline
Serviço de Identidade & IAM plugin & IAM API & Keystone & OAuth, LDAP \\
\hline
Agendamento & Scheduler & Interno & Nova-scheduler & Kubernetes Scheduler \\
\hline
Repositório de Imagens & Interno & Interno & Glance & OpenShift Registry \\
\hline
Cobrança e Faturamento & Interno & CloudStack Usage & Ceilometer & Não embutido \\
\hline
Registro & Interno & Interno & Interno & EFK Stack \\
\hline
\multicolumn{5}{|c|}{\textbf{Suporte}} \\
\hline
Barramento de Mensagens & Interno/RabbitMQ & Interno/RabbitMQ & RabbitMQ & Não aplicável \\
\hline
Banco de Dados & SQLite/MySQL & MySQL & MySQL/Galera/ MariaDB/MongoDB & etcd \\
\hline
Serviço de Transferência & Interno & Interno & Nova Object Store/Cinder & Não aplicável \\
\hline
\multicolumn{5}{|c|}{\textbf{Gerenciamento}} \\
\hline
Gerenciamento de Recursos & Interno & Interno & Nova & Kubernetes \\
\hline
Gerenciamento de Federação & Não disponível & Não disponível & Não disponível & OpenShift Federation \\
\hline
Gerenciamento de Elasticidade & Auto-scaling & Elastic Load Balancing & Elastic Recheck & HPA \\
\hline
Gerenciamento de Usuários/Grupos & Interno & Interno & Interno & RBAC \\
\hline
Definição de SLA & Não disponível & Não disponível & Não disponível & Não embutido \\
\hline
Monitoramento & Probe/SSH/OneGate & Externo & Externo & Prometheus, Grafana \\
\hline
\multicolumn{5}{|c|}{\textbf{Ferramentas de Gerenciamento}} \\
\hline
Ferramentas CLI & OpenNebula CLI & CloudMonkey & OpenStack CLI & OpenShift CLI (oc) \\
\hline
APIs & Nuvem pública e Plugins & Nuvem pública e Plugins & Nuvem pública e Plugins & REST APIs \\
\hline
Painel & Sunstone (Admin UI, User UI) & Admin UI & Horizon (Admin UI) & OpenShift Web Console \\
\hline
Orquestrador & OneFlow & CloudStack Cookbook & Heat & Kubernetes \\
\hline
\multicolumn{5}{|c|}{\textbf{Segurança}} \\
\hline
Autenticação & Basic Auth/OpenNebula Auth/x.509/LDAP & SAML/LDAP & LDAP /Tokens/ X.509/HTTPD & OAuth, LDAP \\
\hline
Autorização & Auth driver & SAML & Keystone & RBAC \\
\hline
Grupos de Segurança & Interno & Interno & Interno & Network Policies \\
\hline
Single Sign-On & Não disponível & Externo & Não disponível & Provedores OAuth \\
\hline
Monitoramento de Segurança & Externo & Externo & Externo & Compliance Operator, Falco \\
\hline
\multicolumn{5}{|c|}{\textbf{Controle}} \\
\hline
Aplicação de SLA & Não disponível & Não disponível & Não disponível & Não embutido \\
\hline
Monitoramento de SLA & Não disponível & Não disponível & Não disponível & Não embutido \\
\hline
Medição & Externo & Plugin de Uso & Ceilometer & OpenShift Metering \\
\hline
Controle de Políticas & Não disponível & Não disponível & Não disponível & Open Policy Agent \\
\hline
Serviço de Notificação & Não disponível & Interno & Não disponível & Não embutido \\
\hline
Orquestração & Interno & Interno & Interno & Kubernetes \\
\hline
\multicolumn{5}{|c|}{\textbf{Serviços de Valor Agregado}} \\
\hline
Zonas de Disponibilidade & Interno & Interno & Interno & Suporta multi-zonas \\
\hline
Alta Disponibilidade & Externo & Externo & Externo & Configurações de HA \\
\hline
Suporte Híbrido & Microsoft Azure & Amazon EC2/IBM & HP Helion/Amazon EC2/IBM & Implantações híbridas \\
\hline
Migração ao Vivo & Interno & Interno & Interno & Migração de aplicações \\
\hline
Suporte à Portabilidade & Não disponível & Não disponível & Não disponível & Portabilidade de contêineres \\
\hline
Contextualização de Imagem & One-context & Não disponível & Não disponível & Source-to-Image (S2I) \\
\hline
Suporte a Aplicações & Não disponível & Não disponível & Não disponível & Implantação via Kubernetes \\
\hline
\end{longtable}

\normalsize % Retorna ao tamanho de fonte normal

Após a análise da tabela \ref{tab:comparacao_ferramentas_nuvem}, percebemos que, de forma geral, o OpenStack apresenta serviços mais modularizados e com maior potencial de integração. Isso torna a solução mais configurável e escalável, o que é especialmente vantajoso, considerando que os serviços necessários em um campus voltado à pesquisa podem variar significativamente. 

\cite{wen2012comparison} apresentam uma comparação detalhada entre as plataformas de gerenciamento de nuvem OpenStack e OpenNebula. Elas são comparadas sob diversos aspectos, incluindo origem, arquitetura, \textit{hipervisor} entre outros detalhes. E destaca o OpenStack em relação à escalabilidade do serviço e os componentes existentes dentro dele, mas finaliza explicando que vai depender de como você vai utilizar esse serviço em nuvem para efetuar a escolha certa.

\cite{shahzadi2017infrastructure} apresentam uma análise comparativa do desempenho de diferentes plataformas de nuvem de código aberto focadas em IaaS. O estudo examina aspectos como arquitetura, modelos de nuvem suportados, compatibilidade com \textit{hypervisors} e linguagens de programação. A comparação destaca o desempenho e a escalabilidade de cada plataforma, oferecendo recomendações para pesquisadores sobre a escolha adequada de ferramentas de validação em ambientes de teste. A análise conclui que o OpenStack demonstra desempenho superior em cenários de teste, especialmente ao comparar implantações com o \textit{Native OpenStack Approach}.

\cite{kumar2014open} apresentam uma comparação abrangente entre as principais plataformas de nuvem de código aberto, incluindo OpenStack, Eucalyptus, CloudStack e OpenNebula. O estudo aborda aspectos como origem, arquitetura, compatibilidade com \textit{hypervisors}, suporte a APIs e capacidade de migração de VMs. A análise destaca o OpenStack como uma solução robusta e escalável, com suporte comunitário amplo e uma arquitetura modular que facilita a integração de diversos serviços. No entanto, conclui que a escolha ideal entre as plataformas depende dos requisitos específicos do ambiente, como suporte a APIs da AWS, integração com sistemas existentes e casos de uso para nuvens públicas, privadas ou híbridas.



\section{Outras Motivações para a Escolha do OpenStack}

Ao considerar uma arquitetura de nuvem privada completa, o OpenShift é comumente utilizado em conjunto com outros serviços de nuvem devido ao seu foco no provisionamento e na orquestração de contêineres. Frequentemente, ele complementa as funcionalidades das soluções de nuvem privada~\citep{IBMWithOpenShift}.

Como todos esses projetos são de código aberto (\textit{open-source}), é importante analisar a atividade das comunidades que os mantêm. Dentre eles, o OpenStack se destaca por possuir uma comunidade extremamente ativa e numerosa.

\begin{table}[htbp]
    \centering
    \caption{Comparação da atividade das comunidades dos projetos open-source}
    \label{tab:comunidade_projetos}
    \begin{tabular}{|l|c|c|}
        \hline
        \textbf{Projeto} & \textbf{Contribuidores} & \textbf{Estrelas} \\ \hline
        OpenStack & 2.640 & 5.300 \\ \hline
        Apache CloudStack & 415 & 2.000 \\ \hline
        OpenNebula & 162 & 1.200 \\ \hline
        OpenShift & 534 & 8.500 \\ \hline
    \end{tabular}
\end{table}

Ao comparar a comunidade do OpenStack com a do OpenNebula, percebe-se que, apesar de o OpenNebula ter sido lançado antes, em 2011, a comunidade do OpenStack já era maior e contava com grandes empresas contribuindo para seu desenvolvimento~\citep{wen2012comparison}.

\subsection{Utilização em Instituições de Ensino}

Ao pesquisar a infraestrutura de grandes instituições de ensino, observa-se que muitas delas utilizam tecnologias de nuvem privada. Abaixo estão alguns exemplos:

\begin{itemize}
    \item \textbf{Harvard}: Utiliza serviços específicos do OpenStack para a criação de um \textit{Cloud Dataverse}~\citep{CloudDataverseSwiftHarvard}.
    \item \textbf{Universidade da Califórnia}: Possui uma nuvem privada com OpenStack que utiliza entre 5.000 núcleos de CPU e 50 TB de armazenamento~\citep{UniversitiesUsingOpenStackOpenMetal}.
    \item \textbf{Cambridge}: Mantém uma nuvem privada com OpenStack com até 2.000 núcleos de CPU e 20 TB de armazenamento~\citep{UniversitiesUsingOpenStackOpenMetal}.
    \item \textbf{UCLouvain}: Utiliza o OpenNebula para fornecer serviços em nuvem para seus alunos~\citep{UCLouvainOpenNebula}.
    \item \textbf{USP}: Adota o Apache CloudStack para prover serviços em nuvem aos seus alunos~\citep{USPCloudStack}.
\end{itemize}

Outro ponto que levou à escolha do OpenStack é sua superioridade em resiliência e escalabilidade quando comparado aos concorrentes~\citep{vogel2016private}. Além disso, o OpenStack apresenta uma fragmentação mais eficiente dos componentes. Cada serviço listado na Tabela \ref{tab:comparacao_ferramentas_nuvem} possui um componente específico dentro do OpenStack, dedicado a garantir seu funcionamento adequado~\citep{vogel2016private}.
