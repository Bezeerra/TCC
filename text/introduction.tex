\chapter{Introdução}
\label{chp:introduction}

O primeiro método de compartilhamento de recursos precedente a computação em nuvem foi o `time sharing' (Tempo Compartilhado), começou entre 1950 e 1960, quando a ideia inicial foi comprar máquinas potentes e manter rotinas de tempo compartilhado, com outros programas acessando a máquina simultaneamente, cada programa recebia um tempo e era capaz de utilizar os recursos da máquina durante o período, implementado no MIT em 1963 e utilizado até 1973~\citep{HistoryOfCloudByIBM}. Em 1961, John McCarthy avançou na ideia e escreveu sobre como a força computacional deveria ser vendida como um serviço público, uma \textit{``commodity''} \citep[History Cloud]{arutyunov2012cloud}, o modelo de pagamento apenas pelo uso, semelhante ao que ocorre com serviços como água e luz, e adicionou que seria uma nova industria muito relevante no futuro~\citep{qian2009cloud}.

Com o avanço da tecnologia, a popularização do acesso à banda larga e a redução dos custos dos computadores residenciais~\citep{qian2009cloud}, surgiu uma nova onda de sites ``.com''. O número de soluções de \textit{SaaS (Software as a Service)} no mercado cresceu exponencialmente~\citep{InternetLiveWebsites}, aumentando significativamente a demanda por infraestrutura capaz de suportar esses serviços. A necessidade de adquirir hardware e montar servidores para hospedar sites, mesmo os considerados `simples', fez com que a ideia de comercializar força computacional se tornasse cada vez mais atrativa. Dessa forma, o conceito de IaaS \textit{(Infrastructure as a Service)} na nuvem foi se desenvolvendo, permitindo que as pessoas pagassem apenas pelo uso da infraestrutura de terceiros conforme suas necessidades, eliminando a necessidade de adquirir todo o hardware e personalizar servidores por conta própria.

A computação em nuvem evoluiu a passos lentos até a chegada da AWS em 2006. A AWS contribuiu significativamente para o aumento na quantidade de pessoas adquirindo infraestruturas no formato IaaS e foi uma das principais difusoras dessa tecnologia, iniciando com a venda de armazenamento como serviço \textbf{S3}~\citep{AWSlauch}. Porém, foi apenas em 2010 que tivemos a entrada de outras grandes empresas no mercado, como o Google Cloud e o Azure. Em 2008 foi criado o LXC (Linux Containers Projects) pela IBM~\citep{HistoryOfContainers}, esse seria a base para a criação de novas tecnologias como o Docker que iriam ser utilizados para criação de microsserviços em nuvem. 

Em 2013, o lançamento da tecnologia Docker~\citep{DockerDocumentation} revolucionou o uso de contêineres, otimizando a gestão de recursos e permitindo a execução leve e isolada de aplicações, o que levou o Google a migrar todos os microsserviços para contêineres~\citep{HistoryOfContainers}. Em 2014, o Google lançou o Kubernetes, uma tecnologia fundamental para a automação e escalabilidade na orquestração de contêineres, impulsionando a adoção de serviços em nuvem e acelerando a migração das infraestruturas empresariais~\citep{CompaniesMigrateToAWS}. Desde então, o uso de serviços em nuvem tem crescido rapidamente entre as empresas [\citep{GrowthCompaniesInCloud_one}; \citep{GrowthCompaniesInCloud_two}].


Embora a computação em nuvem pública tenha democratizado o acesso a infraestruturas escaláveis, sua dependência de provedores externos levanta preocupações estratégicas para organizações com demandas específicas de segurança como bancos. Como alternativa, surgiram as nuvens privadas, controladas internamente pelas próprias empresas, que se apresentam como uma estratégia viável para evitar a dependência de infraestrutura de terceiros, mantendo os benefícios característicos das nuvens públicas. Ao adotar uma nuvem privada, as organizações podem configurar e gerenciar sua própria infraestrutura com as mesmas arquiteturas utilizadas em nuvens públicas, garantindo a flexibilidade necessária para alocar recursos de maneira automática ou personalizada, de acordo com suas necessidades. Além disso, o controle direto sobre dados e segurança, aspectos cruciais para muitas corporações, permanece integralmente no ambiente corporativo, promovendo maior tranquilidade e redução de riscos~\citep{privateCloudAdvantagesDisadvantages}.


Seguindo a ideia dos fornecedores de IaaS, surge um questionamento: como promover algo similar no Instituto Federal de Brasília de forma que os alunos e todos os demais membros da comunidade acadêmica possam utilizar a infraestrutura de maneira simples e escalável? Além disso, como a instituição pode aproveitar ao máximo o hardware remanescente, transformando-o em uma infraestrutura de nuvem que não apenas suporte as necessidades acadêmicas e administrativas, mas também fomente estudos e pesquisas sobre essa tecnologia no campus? A criação de uma nuvem privada mantida pela própria faculdade pode ser a solução, permitindo que recursos ociosos sejam reutilizados de forma eficiente, proporcionando um ambiente seguro e de baixo custo para todos os envolvidos.

\section{Justificativa} 

A evolução da computação em nuvem revolucionou a maneira como recursos computacionais são consumidos, tornando-se uma ferramenta essencial para a competitividade e eficiência de empresas modernas. Atualmente, segundo pesquisas, cerca de 90\% das empresas utilizam algum tipo de serviço em nuvem, sejam eles privados, híbridos ou públicos, com benefícios econômicos significativos que podem representar até um aumento de 11.2\% na receita~\citep{StatisticsCloud, ProfitCloud}. Essa adesão massiva é impulsionada pela flexibilidade, escalabilidade e redução de custos proporcionada pela computação em nuvem.

Apesar das vantagens, a dependência da infraestrutura de terceiros ao utilizar nuvens públicas apresenta desafios, especialmente em termos de segurança e controle sobre dados sensíveis e custos dependendo da aplicação~\citep{privateCloudAdvantagesDisadvantages}. Por isso, a adoção de nuvens privadas, que permite às empresas configurar e gerenciar sua própria infraestrutura, tem ganhado destaque como uma alternativa estratégica. Neste contexto, surge a necessidade de analisar como instituições acadêmicas podem implementar nuvens privadas para aproveitar ao máximo seus recursos computacionais e oferecer aos alunos um ambiente prático e seguro para aprendizagem e pesquisa.

Diante desse cenário, a presente pesquisa busca justificar a implantação e implementar uma nuvem privada em um ambiente acadêmico, com foco em proporcionar um ambiente seguro e de baixo custo, aproveitando hardware ocioso e oferecendo serviços de maneira simples e escalável para os alunos. Isso possibilita não apenas o uso eficiente da infraestrutura disponível, mas também fomenta estudos e pesquisas na área de computação em nuvem, capacitando os estudantes e promovendo o desenvolvimento de novas tecnologias.

\section{Objetivos}

O objetivo geral deste projeto é propor uma arquitetura para implementação de uma nuvem privada no campus do Instituto Federal de Brasília (IFB), utilizando os recursos computacionais já existentes e gerida pela própria instituição. Essa implementação servirá como prova de conceito e permitirá que alunos, professores e funcionários tenham acesso a uma plataforma de computação escalável, segura e eficiente, capaz de atender às diversas demandas acadêmicas e administrativas.

Com a implementação desse projeto espera-se alcançar os seguintes benefícios:

\begin{itemize}
    \item \textbf{Otimização do Uso de Recursos:} Aproveitar o hardware remanescente e ocioso da instituição, transformando-o em uma infraestrutura de nuvem capaz de suportar a execução de aplicações acadêmicas, projetos de pesquisa e atividades administrativas, reduzindo custos operacionais e aumentando a eficiência.

    \item \textbf{Fomento à Pesquisa e Inovação:} Criar um ambiente que facilite o estudo e a pesquisa sobre tecnologias de computação em nuvem, permitindo que os alunos desenvolvam projetos práticos utilizando a infraestrutura de nuvem privada, alinhando a teoria aprendida em sala de aula com a prática.

    \item \textbf{Segurança e Controle de Dados:} Garantir que os dados sensíveis da instituição sejam mantidos em um ambiente seguro e controlado, minimizando os riscos associados ao uso de nuvens públicas e atendendo às exigências de conformidade com políticas de segurança da informação.

    \item \textbf{Facilitação do Acesso a Serviços de Computação:} Proporcionar um ambiente de fácil acesso e uso para todos os membros da comunidade acadêmica, com a possibilidade de escalabilidade conforme as necessidades individuais de cada usuário, garantindo um serviço confiável e de alta disponibilidade.

    \item \textbf{Redução de Dependência de Infraestruturas Externas:} Diminuir a dependência da instituição em relação a serviços de nuvem pública, promovendo a autossuficiência e a autonomia tecnológica, o que pode resultar em economia financeira e maior controle sobre os recursos utilizados.

    \item \textbf{Capacitação de Alunos e Colaboradores:} Oferecer treinamentos sobre o uso da infraestrutura de nuvem privada, capacitando alunos e colaboradores a utilizarem essa tecnologia de forma eficiente, fomentando o desenvolvimento de competências que são altamente demandadas no mercado de trabalho atual.
\end{itemize}

